\documentclass{article}
\usepackage{amsmath, amssymb, amsthm}
\usepackage{graphicx}
\usepackage{listings} % Для кода на Python
\usepackage{xcolor} % Цветной код
\usepackage[utf8]{inputenc}
\usepackage[russian]{babel}
\usepackage{booktabs}
\usepackage{float}
\usepackage{caption}

\lstset{
    language=Python,
    basicstyle=\ttfamily\small,
    keywordstyle=\color{blue},
    commentstyle=\color{gray},
    stringstyle=\color{red},
    breaklines=true
}

\begin{document}

\section*{Расчетно графическая работа по функциональному анализу. Задание №\,1}

Выполнил студент группы М80-308Б-22 МАИ \textit{Цирулев Николай}.

\section{Задание}
Задание I
Докажите, что приведённое ниже отображение $T: C[0 ; 1] \rightarrow C[0 ; 1]$ (либо его степень) является сжимающим. Определите число итераций, необходимое для поиска неподвижной точки этого отображения с точностью $\varepsilon \in\left\{10^{-1}, 10^{-2}, 10^{-3}\right\}$ с помощью метода сжимающих отображений. С помощью вычислительной техники постройте график функции, являющейся неподвижной точкой отображения $T$. Проверьте результаты при различных значениях $\varepsilon$ и различных начальных приближениях в методе сжимающих отображений. Продемонстрируйте несколько графиков, получающихся при промежуточных вычислениях.
\\
Номер по списку 22, $k = 8, l = 5$\\
Вариант: \\
T(x)= \begin{cases}\frac{1}{1+k} x(3 t)-\frac{l}{2}, & 0 \leqslant t \leqslant \frac{1}{3} ; \\ f(t), & \frac{1}{3}<t<\frac{2}{3} ;  \\ \frac{1}{1+k} x(3 t-2), & \frac{2}{3} \leqslant t \leqslant 1,\end{cases}
 \text { где } f(t)-\text { аффинная функция такая, что } T(x)-\text { непрерывная функция;}


\section{Решение}
\subsection{Доказательство и поиск коэффициента сжатия}}

Докажем, что заданное отображение \( T: C[0,1] \to C[0,1] \) является сжимающим.

Пространство \( (C[0,1], \rho) \) рассматривается с метрикой:
\[
\rho(x,y) = \max_{t \in [0,1]} |x(t) - y(t)|.
\]

Определение оператора \( T(x) \):

\[
T(x)(t) =
\begin{cases}
\frac{1}{9} x(3t) - \frac{5}{2}, & 0 \leq t \leq \frac{1}{3}; \\
f(t), & \frac{1}{3} < t < \frac{2}{3}; \\
\frac{1}{9} x(3t-2), & \frac{2}{3} \leq t \leq 1.
\end{cases}
\]

По определению отображение \( T(x) \) является сжимающим, если $\exists$ \( \alpha \in [0,1) \) :

\[
\rho(T(x), T(y)) \leq \alpha \cdot \rho(x, y), \quad \forall x, y \in C[0,1].
\]

Рассмотрим три области отдельно.

\paragraph{1. Область \( 0 \leq t \leq \frac{1}{3} \)}  
\[
T(x)(t) = \frac{1}{9} x(3t) - \frac{5}{2},
\]
\[
T(y)(t) = \frac{1}{9} y(3t) - \frac{5}{2}.
\]
Тогда:

\[
|T(x)(t) - T(y)(t)| = \left| \frac{1}{9} x(3t) - \frac{5}{2} - \left(\frac{1}{9} y(3t) - \frac{5}{2} \right) \right| = \frac{1}{9} |x(3t) - y(3t)|.
\]

Так как \( |x(3t) - y(3t)| \leq \rho(x, y) \), то если $s = 3t, t \in [0; \frac{1}{3}] =>  s \in [0; 1]$, тогда:

\[
\max_{s \in [0,1]} |T(x)(s) - T(y)(s)| \leq \frac{1}{9} \rho(x, y).
\]

\paragraph{2. Область \( \frac{2}{3} \leq t \leq 1 \)}  
\[
T(x)(t) = \frac{1}{9} x(3t - 2),
\]
\[
T(y)(t) = \frac{1}{9} y(3t - 2).
\]
Аналогично если $s = 3t - 2, t \in [\frac{2}{3}; 1] =>  s \in [0; 1]$, тогда:

\[
|T(x)(t) - T(y)(t)| = \frac{1}{9} |x(3t - 2) - y(3t - 2)| \leq \frac{1}{9} \rho(x, y).
\]

\paragraph{3. Область \( \frac{1}{3} < t < \frac{2}{3} \)}  
Функция \( T(x) \) здесь аффинная и непрерывная. Из конструкции \( T(x) \) видно, что на этом интервале максимум отклонений также не превосходит \( \frac{1}{9} \rho(x, y) \).

\paragraph{Вывод}  
Объединяя оценки для всех трёх областей, получаем:

\[
\rho(T(x), T(y)) = \max_{t \in [0,1]} |T(x)(t) - T(y)(t)| \leq \frac{1}{9} \rho(x, y).
\]

Следовательно, \( T \) является сжимающим отображением с коэффициентом \( \alpha = \frac{1}{9} \).
\\
\\
\subsection*{Определение числа итераций и построение графиков}

По теореме Банаха, так, как метрическое пространство полное, то \(\exists! неподвижная точка x_*\) отображения T. Тогда определим число итераций для поиска неподвижной точки и построим графики.

Сначала оценим $f(t)$: \\
Поскольку $T(x)$ должно быть непрерывным по условию, функция $f(t)$ в интервале $\left(\frac{1}{3}, \frac{2}{3}\right)$ должна быть линейной и соединять значения $T(x)$ на границах $t=\frac{1}{3}$ и $t=\frac{2}{3}$.
То есть, f(t)$ можно записать в виде:
$$
f(t)=a t+b
$$

Найдем коэффициенты $a$ и $b$. Пусть $A=T(x)\left(\frac{1}{3}\right)$ и $B=T(x)\left(\frac{2}{3}\right)$, тогда система уравнений:
$$
\begin{aligned}
& A=a \cdot \frac{1}{3}+b \\
& B=a \cdot \frac{2}{3}+b
\end{aligned}
$$

Решая систему, получаем:
$$
\begin{gathered}
a=\frac{B-A}{\frac{2}{3}-\frac{1}{3}}=3(B-A) \\
b=A-a \cdot \frac{1}{3}=A-(B-A)=2 A-B .
\end{gathered}
$$

Тогда:
$$
f(t)=3(B-A) t+(2 A-B)
$$

Теперь оценим число итераций:
\[
n > \frac{\ln \left(\varepsilon \cdot (1 - \alpha) / \rho(x_1 - x_0) \right)}{\ln \alpha}.
\]
Где:
- \( \alpha = \frac{1}{9} \),
- \( x_0(t) = t \),
- \( \rho(x_1 - x_0) = \frac{5}{2} \).

Для различных \( \varepsilon \):

\begin{tabular}{|c|c|}
\hline
\( \varepsilon \) & \( n \) \\
\hline
\( 10^{-1} \) & 2 \\
\( 10^{-2} \) & 3 \\
\( 10^{-3} \) & 4 \\
\hline
\end{tabular}
\\

\section{Код, графики и выводы}

\subsection{Код на Python}

\begin{lstlisting}
import numpy as np
import matplotlib.pyplot as plt

# Функция T
def T(x, t, k=8, l=5):
    if t <= 1/3:
        return (1 / (1 + k)) * x(3 * t) - l / 2
    elif t >= 2/3:
        return (1 / (1 + k)) * x(3 * t - 2)
    else:
        # Определяем граничные значения
        A = T(x, 1/3)
        B = T(x, 2/3)
        return 3 * (B - A) * t + (2 * A - B)

# Итерации функции T
def iterate_T(x0, num_iter=10, num_points=1000):
    t_values = np.linspace(0, 1, num_points)
    x_values = x0(t_values)
    history = [x_values.copy()]
    
    for _ in range(num_iter):
        x_values_new = np.array([T(lambda t: np.interp(t, t_values, x_values), t) for t in t_values])
        history.append(x_values_new.copy())
        x_values = x_values_new
    
    return t_values, history

# Функция для вычисления ошибки
def compute_error(history, alpha=1/9):
    errors = []
    rho_0 = np.max(np.abs(history[0] - history[1]))
    for n in range(1, len(history)):
        error = (alpha ** n) * rho_0 / (1 - alpha)
        errors.append(error)
    return errors

# Начальные приближения с LaTeX-записями
x0_list = [
    (lambda t: t, r'$x_0(t) = t$'),
    (lambda t: np.sin(2 * np.pi * t), r'$x_0(t) = \sin(2\pi t)$'),
    (lambda t: np.exp(-t), r'$x_0(t) = e^{-t}$')
]

# Построение графиков
for x0, formula in x0_list:
    t_values, history = iterate_T(x0, num_iter=5)
    errors = compute_error(history)
    
    plt.figure(figsize=(8, 5))
    for i, x_values in enumerate(history):
        plt.plot(t_values, x_values, label=f'Итерация {i}')
    
    plt.title(f'Итерации для {formula}', fontsize=14)
    plt.xlabel(r'$t$', fontsize=12)
    plt.ylabel(r'$x(t)$', fontsize=12)
    plt.legend()
    plt.grid(True)
    plt.show()

    # Вывод таблицы значений x и eps для каждой итерации
    print(f'Таблица значений для {formula}:\n')
    print(f'{"Итерация":^10} {"x(0.5)":^15} {"Ошибка eps":^15}')
    print('-' * 40)
    
    for i, x_values in enumerate(history):
        x_mid = x_values[len(x_values) // 2]
        eps = errors[i - 1] if i > 0 else 0  
        print(f'{i:^10} {x_mid:^15.5f} {eps:^15.5f}')
    
    print('\n' + '=' * 40 + '\n')

\end{lstlisting}

\subsection{Графики}

\begin{figure}[!ht] 
    \centering
    \includegraphics[width=0.5\textwidth]{11.png}
\end{figure}

\begin{figure}[!ht] 
    \centering
    \includegraphics[width=0.5\textwidth]{22.png}
\end{figure}

\begin{figure}[!ht] 
    \centering
    \includegraphics[width=0.5\textwidth]{33.png}
\end{figure}


\begin{table}[H]
    \centering
    \small
    \setlength{\tabcolsep}{10pt}
    \renewcommand{\arraystretch}{1.2}
    \caption{Значения для $x_0(t) = t$}
    \begin{tabular}{c c c}
        \toprule
        Итерация & $x(0.5)$ & Ошибка $\varepsilon$ \\
        \midrule
        0 & 0.50050 & 0.00000 \\
        1 & -1.19086 & 0.34028 \\
        2 & -1.37940 & 0.03781 \\
        3 & -1.40035 & 0.00420 \\
        4 & -1.40267 & 0.00047 \\
        5 & -1.40293 & 0.00005 \\
        \bottomrule
    \end{tabular}
\end{table}

\begin{table}[H]
    \centering
    \small
    \setlength{\tabcolsep}{10pt}
    \renewcommand{\arraystretch}{1.2}
    \caption{Значения для $x_0(t) = \sin(2\pi t)$}
    \begin{tabular}{c c c}
        \toprule
        Итерация & $x(0.5)$ & Ошибка $\varepsilon$ \\
        \midrule
        0 & -0.00314 & 0.00000 \\
        1 & -1.24625 & 0.45139 \\
        2 & -1.38555 & 0.05015 \\
        3 & -1.40103 & 0.00557 \\
        4 & -1.40275 & 0.00062 \\
        5 & -1.40294 & 0.00007 \\
        \bottomrule
    \end{tabular}
\end{table}

\begin{table}[H]
    \centering
    \small
    \setlength{\tabcolsep}{10pt}
    \renewcommand{\arraystretch}{1.2}
    \caption{Значения для $x_0(t) = e^{-t}$}
    \begin{tabular}{c c c}
        \toprule
        Итерация & $x(0.5)$ & Ошибка $\varepsilon$ \\
        \midrule
        0 & 0.60623 & 0.00000 \\
        1 & -1.17015 & 0.42361 \\
        2 & -1.37710 & 0.04707 \\
        3 & -1.40009 & 0.00523 \\
        4 & -1.40265 & 0.00058 \\
        5 & -1.40293 & 0.00006 \\
        \bottomrule
    \end{tabular}
\end{table}

\newpage
\subsection{Выводы}

Мы доказали, что \( T \) — сжимающее отображение строго по определению, нашли коэффициент сжатия, вычислили число итераций для различных \( \varepsilon \) и реализовали метод сжимающих отображений на Python с графической интерпретацией.

\end{document}
