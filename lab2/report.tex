\documentclass{article}
\usepackage{amsmath, amssymb, amsthm}
\usepackage{graphicx}
\usepackage{listings}
\usepackage{xcolor}
\usepackage[utf8]{inputenc}
\usepackage[russian]{babel}
\usepackage{booktabs}
\usepackage{float}
\usepackage{caption}
\usepackage[T1,T2A]{fontenc}
\usepackage{minted}

\begin{document}

\section*{Расчетно графическая работа по функциональному анализу. Задание №\,2}

Выполнил студент группы М80-308Б-22 МАИ \textit{Цирулев Николай}.

\section{Задание}
Проведите ортогонализацию системы функций $x_n(t)=t^{n-1}$ в пространстве квадратично суммируемых функций относительно склярного произведения $\langle x, y\rangle=\int_a^b x(t) y(t) f(t) d t$. Найдите приближение функции $y$ частичной суммой ряда Фурье, обеспечивающее среднеквадратичную точность разложения $\varepsilon \in\left\{10^{-1}, 10^{-2}, 10^{-3}\right\}$ (при достаточных вычислительных ресурсах). Постройте график функции $y(t)$ и его приближения частичными суммами ряда Фурье. Продемонстрируйте несколько графиков, получающихся при промежуточных вычислениях.

\\
Номер по списку 22, $k = 8, l = 5$\\
Вариант: \\
$[a, b]=\left[0 ; 0,8+\frac{k}{10}\right], f(t)=\left(\frac{2 l}{5}-t\right) t, y(t)=\cos (2 t)$;


\section{Решение}
\begin{minted}{python}
import numpy as np
import matplotlib.pyplot as plt
from scipy import integrate
import math

k = 8
l = 5
a = 0
b = 0.8 + k/10

iter_count = 7

def f(t):
    return (2 * l / 5 - t) * t

def y(t):
    return np.cos(2 * t)

def create_basis_function(n):
    return lambda t: t**(n-1)

def scalar_product(func1, func2):
    integrand = lambda t: func1(t) * func2(t) * f(t)
    result, _ = integrate.quad(integrand, a, b)
    return result

"""Применим процесс Грамма-Шмидта для ортогонализации"""
def gram_schmidt(n_max):
    ortho_polys = []
    for n in range(1, n_max + 1):
        phi_n = lambda t, n=n: create_basis_function(n)(t)
        for phi_k in ortho_polys:
            coeff = scalar_product(phi_n, phi_k) / scalar_product(phi_k, phi_k)
            phi_n = lambda t, phi_n=phi_n, phi_k=phi_k, coeff=coeff: phi_n(t) - coeff * phi_k(t)
        norm = np.sqrt(scalar_product(phi_n, phi_n))
        ortho_polys.append(lambda t, phi_n=phi_n, norm=norm: phi_n(t) / norm)
    return ortho_polys
"""Вычислим частичные суммы ряда Фурье"""
def fourier_approximation(orthonormal_functions, x_func, n):

    coefficients = []
    for e_i in orthonormal_functions[:n]:
        c_i = scalar_product(x_func, e_i)
        coefficients.append(c_i)
    
    def approximation(t):
        result = 0
        for c_i, e_i in zip(coefficients, orthonormal_functions[:n]):
            result += c_i * e_i(t)
        return result
    
    return approximation
"""Вычислим длину ошибки проектирования"""
def calculate_error(original, approximation):
    error_func = lambda t: (original(t) - approximation(t))**2
    error, _ = integrate.quad(error_func, a, b)
    return math.sqrt(error)

t_points = np.linspace(a, b, 500)

orthogonal_functions = gram_schmidt(iter_count)

plt.figure(figsize=(12, 6))
plt.plot(t_points, [y(t) for t in t_points], 'b-', label='$y(t) = \cos(2t)$')

errors = []
approximations = []
for n in range(1, iter_count):
    approx = fourier_approximation(orthogonal_functions, y, n)
    error = calculate_error(y, approx)
    errors.append(error)
    approximations.append(approx)
    
    plt.plot(t_points, [approx(t) for t in t_points], '--', 
            label=f'Приближение (n = {n}), ошибка = {error:.6f}')

plt.xlabel('t')
plt.ylabel('y')
plt.legend()
plt.grid(True)
plt.show()

print("\nДлина ошибки проектирования для различных n:")
for n, error in enumerate(errors, 1):
    print(f"n = {n}: {error:.6f}")


for eps in [1e-1, 1e-2, 1e-3]:
    n_required = next((n for n, err in enumerate(errors, 1) if err < eps), len(errors))
    print(f"\nДля ε = {eps} n = {n_required}")
\end{minted}
\section{Вывод}

\begin{figure}[!ht] 
    \centering
    \includegraphics[width=1\textwidth]{Без имени.png}
\end{figure}

Длина ошибки проектирования для различных n: \\
n = 1: 0.929740 \\
n = 2: 0.117902 \\
n = 3: 0.140000 \\
n = 4: 0.004040 \\
n = 5: 0.005475 \\
n = 6: 0.000064 \\
 \\
Как мы видим: \\
Для $\epsilon$ = 0.1 минимальное n = 4 \\
Для $\epsilon$ = 0.01 минимальное n = 4 \\
Для $\epsilon$ = 0.001 минимальное n = 6
\end{document}
