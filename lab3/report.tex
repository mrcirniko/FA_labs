\documentclass{article}
\usepackage{amsmath, amssymb, amsthm}
\usepackage{graphicx}
\usepackage{listings}
\usepackage{xcolor}
\usepackage[utf8]{inputenc}
\usepackage[russian]{babel}
\usepackage{booktabs}
\usepackage{float}
\usepackage{caption}
\usepackage[T1,T2A]{fontenc}
\usepackage{minted}

\begin{document}

\section*{Расчетно графическая работа по функциональному анализу. Задание №\,3}

Выполнил студент группы М80-308Б-22 МАИ \textit{Цирулев Николай}.

\section{Задание}
Вычислите интеграл Лебега-Стилтьеса $\int_{[a, b]} f(x) d F(x)$.
\\
Номер по списку 22, $k = 8, l = 5$\\
Вариант: \\
$[a, b]=[-k, 5 l] ; f(x)=2 \cos k x+2 \chi\left(3 x-\frac{l}{5}\right)-x^2,$\\
$ F(x)=e^x+2 \chi(x+1)+2 \chi(4 x-k)+x^3$;

\section{Решение}

Функция $F(x)$ порождает меру Лебега–Стилтьеса $\mu_F$, состоящую из:

\begin{itemize}
    \item Абсолютно непрерывной части: $(e^x + 3x^2) dx$,
    \item Скачков в точках $x = -1$ и $x = 2$, с величинами скачков, равными $2$.
\end{itemize}

Тогда интеграл записывается как:
\[
\int_{[-8, 25]} f(x) \, d\mu_F(x) =
\int_{[-8, 25]} f(x)(e^x + 3x^2) dx + f(-1) \cdot \mu_F(\{-1\}) + f(2) \cdot \mu_F(\{2\}).
\]

Подставляя значения:
\[
f(-1) = 2 \cos(8) - 1,\quad f(2) = 2 \cos(16) - 2,\quad \mu_F(\{-1\}) = \mu_F(\{2\}) = 2,
\]
получаем:
\[
\boxed{
\int_{[-8, 25]} f(x) \, d\mu_F(x) =
\int_{[-8, 25]} f(x)(e^x + 3x^2) dx + 2(2 \cos(8) - 1) + 2(2 \cos(16) - 2)
}
\]

\section{Численное вычисление}
Найдём численное значение полученного интеграла с помощью функции \texttt{quad} из модуля \texttt{scipy.integrate} на языке Python:
\subsection{Код на Python}

\begin{lstlisting}[language=Python, basicstyle=\ttfamily\small, frame=single]
import numpy as np
from scipy.integrate import quad

def f(x):
    if x < 1/3:
        return 2 * np.cos(8 * x) - x**2
    else:
        return 2 * np.cos(8 * x) + 2 - x**2

def F_ac_prime(x):
    return np.exp(x) + 3 * x**2

def integrand(x):
    return f(x) * F_ac_prime(x)

I_continuous, _ = quad(integrand, -8, 25)

f_minus1 = 2 * np.cos(8) - 1
f_2 = 2 * np.cos(16) - 2
I_atoms = 2 * f_minus1 + 2 * f_2

I_total = I_continuous + I_atoms

print(f"{I_total:.6f}")
\end{lstlisting}

Вывод:
\begin{verbatim}
-41417222153999.585938
\end{verbatim}

\section{Ответ}
\[
\int_{[-8, 25]} f(x) \, dF(x) =
\int_{[-8, 25]} f(x)(e^x + 3x^2) dx + 2(2 \cos(8) - 1) + 2(2 \cos(16) - 2)
\]

\[
\int_{[-8, 25]} f(x) \, dF(x) \approx -41417222153999.585938
\]

\end{document}
